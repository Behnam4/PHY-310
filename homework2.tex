\documentclass[fleqn]{article}
\oddsidemargin 0.0in
\textwidth 6.0in
\thispagestyle{empty}
\usepackage{import}
\usepackage{amsmath}
\usepackage{graphicx}
\usepackage{flexisym}
\usepackage{amssymb}
\usepackage{bigints} 
\usepackage[english]{babel}
\usepackage[utf8x]{inputenc}
\usepackage{float}
\usepackage[colorinlistoftodos]{todonotes}

\definecolor{hwColor}{HTML}{AD53BA}

\begin{document}

  \begin{titlepage}

    \newcommand{\HRule}{\rule{\linewidth}{0.5mm}}

    \center


    \textsc{\LARGE Arizona State University}\\[1.5cm]

    \textsc{\LARGE Classical Parts/Field/Matter I}\\[1.5cm]


    \begin{figure}
      \includegraphics[width=\linewidth]{asu.png}
    \end{figure}


    \HRule \\[0.4cm]
    { \huge \bfseries Homework Two}\\[0.4cm] 
    \HRule \\[1.5cm]

    \textbf{Behnam Amiri}

    \bigbreak

    \textbf{Prof: Maulik Parikh}

    \bigbreak


    \textbf{{\large \today}\\[2cm]}

    \vfill

  \end{titlepage}

  \begin{enumerate}
    \item A block of mass $m$ is lifted vertically up to height $h$ (against gravity), then moved horizontally by a distance b, then lowered vertically all the
    way back down, then moved horizontally back to its starting point. Calculate the work done on the block on each of the four segments of
    its quadrilateral path.


    \item A cat of mass $m$ jumps onto the outside edge of a slowly spinning ceiling fan. The fan has radius $R$, moment of inertia $I$, and initial
    angular velocity $\omega_0$. What is its new angular velocity?
    


    \item Consider a rocket in outer space where there is no significant gravitational field. Suppose the rocket is initially at rest and uniformly
    accelerates with acceleration $a$ until it reaches a speed $v$. Let $m$ be the initial mass of the rocket. How much work is done by the rocket’s
    engine?
    


    \item A lunar lander on a rescue mission hovers just above the surface of
    the moon. Suppose the lander’s exhaust comes out at 2000 m/s. The
    moon’s gravity is roughly g/6. If the mission can spare a total of only $20\%$ of the total mass in fuel, for how long can the lander hover?
    



    \item A small coin rests on top of a smooth ball. Suppose the coin is given a
    small push (an infinitesimal amount will suffice). If the ball has radius $R$, what is the height measured from the ground at which the coin will
    fall off the ball? (Think about the normal force.) You can neglect
    friction.



    \item  Let $\overrightarrow{F}= (\alpha y, \beta x, 0)$ be a force field. For what values of the constants
    $\alpha$ and $\beta$ is $\overrightarrow{F}$ conservative? For those cases, determine the potential energy $U(\overrightarrow{r})$
    with respect to the reference point $r_0=(1,1,1)$. Verify that $\overrightarrow{F}=-\overrightarrow{\nabla} U$



    \item Suppose there are $N$ billiard balls constrained to move along a straight
    line. Each of the $N$ balls has an arbitrary unknown velocity either to
    the left or to the right. If all collisions between the balls are elastic,
    what is the maximum number of collisions that can take place?
    


    \item A flat piece of cardboard with uniform mass density $\sigma$ (where $\sigma$ is mass
    per area, not per volume) is bounded by the following curves: the $\hat{x}$-axis, the line y = a (where a > 0) and the curve $y=\dfrac{b}{a^2}x^2$ 
    where $b>0$ (this curve is the parabola that connects the origin to the point (a, b)).
    Calculate the moment of inertia of the piece of cardboard when spun
    about (i) the $\hat{x}$-axis, (ii) the $\hat{y}$-axis.



    \item A car going down a 5 degree slope slams the brakes but still rear-ends a
    stalled car in front. The skid marks extend 30 meters from the point of
    collision. The coefficient of kinetic friction between the road and this
    brand of tires is known to be $\mu=0.5$. The speed limit on the road is
    $25 mph$. Was the car speeding? Explain.



    \item An elementary particle of mass M, at rest in the lab frame, spontaneously splits into two particles, of which one is directly detected and
    the other is not. The detected particle is measured to have energy $E^{\prime}$and momentum $p^{\prime}$. Using the relativistic formula
    $E^2=p^2 c^2+m^2c^4$, find the mass of the unobserved particle. (This is actually similar in
    spirit to how the existence of the neutrino was first inferred by Pauli.)

     
  \end{enumerate}

\end{document}
