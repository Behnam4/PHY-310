\documentclass[fleqn]{article}
\oddsidemargin 0.0in
\textwidth 6.0in
\thispagestyle{empty}
\usepackage{import}
\usepackage{amsmath}
\usepackage{graphicx}
\usepackage{bigints} 
\usepackage[english]{babel}
\usepackage[utf8x]{inputenc}
\usepackage{float}
\usepackage[colorinlistoftodos]{todonotes}

\definecolor{hwColor}{HTML}{AD53BA}

\begin{document}

  \begin{titlepage}

    \newcommand{\HRule}{\rule{\linewidth}{0.5mm}}

    \center



    \textsc{\LARGE Arizona State University}\\[1.5cm]

    \textsc{\LARGE Classical Parts/Field/Matter I }\\[1.5cm]


    \begin{figure}
      \includegraphics[width=\linewidth]{asu.png}
    \end{figure}


    \HRule \\[0.4cm]
    { \huge \bfseries Homework One}\\[0.4cm] 
    \HRule \\[1.5cm]

    \textbf{Behnam Amiri}

    \bigbreak

    \textbf{Prof: Maulik Parikh}

    \bigbreak


    \textbf{{\large \today}\\[2cm]}

    \vfill

  \end{titlepage}

  \begin{enumerate}
    \item \textbf{Mass and weight}
    \begin{enumerate}
      \item Consider a particle of mass m and electric charge q, subject to an electric field E. What is its acceleration, a?

      \textcolor{hwColor}{
        The actual force on a particle with charge $q$ is given by $F_e=qE$, then to find the acceleration of the particle
        we can use the following formula: \\
        \\
        $
          \overrightarrow{F}=m\overrightarrow{a} \rightarrow \overrightarrow{a}=\dfrac{F_e}{m}=\dfrac{q\overrightarrow{E}}{m}
        $
      }

      \item Consider pushing one side of a cube-shaped block of mass m (with sides of area A) by applying a uniform pressure p (ignore friction).
      What is the block’s acceleration, a?

        \textcolor{hwColor}{
          $
            \overrightarrow{P}=\dfrac{\overrightarrow{F}}{A} \rightarrow \overrightarrow{F}=A\overrightarrow{P} \\
            \\
            \\
            \Longrightarrow \overrightarrow{a}=\dfrac{A}{m}\overrightarrow{P}
          $
        }

      \item Now consider a particle of mass m in a gravitational field g. What is its acceleration? What is unusual about acceleration due to
      gravity, compared with your earlier answers? (Einstein used this observation as a starting point for his theory of gravity.)

        \textcolor{hwColor}{
          Acceleration of a particle of mass $m$ in a gravitational field $g$ is $\overrightarrow{a}=\overrightarrow{g}$. \\
          \\
          Acceleration is a change in velocity, and velocity, in turn, is a measure of the speed and direction of motion. 
          Gravity causes an object to fall toward the ground at a faster and faster velocity the longer the object falls.
          In comparison to my earlier answers, here acceleration does \textbf{NOT} depend on mass which is not intuitive.
        }


    \end{enumerate}
    
    \item \textbf{Newton’s first law (or why you should wear a seat belt in the backseat)}
    \begin{enumerate}
      \item Write down a formula for the average deceleration if a body moving initially at speed $v_0$ comes to a halt within a distance x.
      \item Suppose a car going at 45 mph comes to an abrupt halt (crashes into a wall, say). If a passenger in the back is not wearing a seat
      belt, they will go flying into the front seat at the original speed $v_0$ (Newton’s first law) and then come to a halt with the front
      seat of the car. Suppose because of crumpling of the car that the front seat comes to a halt within 50 cm. What is the deceleration
      (expressed in g’s) of the passenger in the back seat?

      \item The main effect of a seat belt is to provide additional distance
      over which deceleration takes place, so that the passenger is decelerating even while hurtling towards the front seat. If the front
      seat is 80 cm away from the passenger (and assuming again a 50 cm stopping distance for the front seat) calculate the passenger’s
      deceleration in g’s again.

      It is the deceleration, not the substance with which you have an impact,
      that kills you (which is why jumping into water from a great height can
      be fatal). Your own internal organs are also subject to Newton’s first
      law, of course. A large acceleration or deceleration can overcome the
      tension of the connective tisues binding internal organs to each other...
      you can imagine the rest. Note also from the first part of the question
      that deceleration goes quadratically with speed. Remember all this
      while driving!
      For more on the perils of deceleration, see S. Cooper in
      https://www.youtube.com/watch?v=hTujOtTqSsQ

    \end{enumerate}

    \item Consider a closed system of three particles. (Assume the particles
    lie on a straight line, for simplicity.) Suppose that the potential energy of the system depends only on the separation of the particles i.e.
    $V(x_1, x_2, x_3)=V(x_1-x_2, x_1-x_3, x_2-x_3)$.
    \begin{enumerate}
      \item  As an example, write down the potential energy if the three particles are interacting gravitationally (e.g. Earth-Moon-Sun), and
      check that it takes the form above.

      \textcolor{hwColor}{
        Potential energy $(V)$ of gravity between two objects: $V=G\dfrac{m_1 ~ m_2}{r}$. For example 
        the potential energy between Earth-Moon-Sun is as the following: \\
        \\
        $
          \begin{cases}
            S \rightarrow Sun \\
            M \rightarrow Moon \\
            E \rightarrow Earth \\
          \end{cases} \\
          \\
          V_{total}=V_{SM}+V_{EM}+V_{ES} \\
          \\
          =G\dfrac{m_S ~ m_M}{x_S-x_M}+G\dfrac{m_E ~ m_M}{x_E-x_M}+G\dfrac{m_E ~ m_S}{x_E-x_S}=G\left(\dfrac{m_S ~ m_M}{x_S-x_M}+\dfrac{m_E ~ m_M}{x_E-x_M}+\dfrac{m_E ~ m_S}{x_E-x_S}\right) \\
          \\
          V_{total}(x_S, x_M, x_E)=V(x_S-x_M, x_E-x_M, x_E-x_S)
        $
      }

      \item Prove that Newton’s third law holds for the system. Remember
      that the force on the ith particle is $-\dfrac{\partial }{\partial x_i}V$

      \textcolor{hwColor}{
        $F$ in the definition of potential energy is the force exerted by the force field, e.g., gravity, spring force, etc. 
        The potential energy U is equal to the work you must do against that force to move an object from the $U=0$ reference point to the position $r$.
        The force you must exert to move it must be equal but oppositely directed, and that is why the below formula has a negative in front of it. \\
        \\
        $
          V=-\bigints_{ref}^{x} \overrightarrow{F} dx \Rightarrow \overrightarrow{F(x)}=\dfrac{dV}{dx} \\ \\
        $
        Let's calculate the force between Earth and Sun. \\
        \\
        $
          \overrightarrow{F_{ES}}=-\overrightarrow{\nabla}V_{ES}=-\dfrac{\partial}{\partial x}\left(G\dfrac{m_E ~ m_S}{x_E-x_S}\right)=-G(m_E ~ m_S)\dfrac{0-1}{(x_E-x_S)^2} ~ \hat{x} \\
          \\
          \overrightarrow{F_{ES}}=G\dfrac{m_E ~ m_S}{(x_E-x_S)^2} ~ \hat{x} \\
        $
      }

      \textcolor{hwColor}{
        \rule{15cm}{1pt}
      }

      \textcolor{hwColor}{
        Now the force between Sun and Moon. \\ \\
        $
          \overrightarrow{F_{SM}}=-\overrightarrow{\nabla}V_{SM}=-\dfrac{\partial}{\partial x}\left(G\dfrac{m_S ~ m_M}{x_S-x_M}\right)=-G(m_S ~ m_M)\dfrac{0-1}{(x_S-x_M)^2} ~ \hat{x} \\
          \\
          \overrightarrow{F_{SM}}=G\dfrac{m_S ~ m_M}{(x_S-x_M)^2} ~ \hat{x} \\
        $
      }

      \textcolor{hwColor}{
        \rule{15cm}{1pt}
      }

      \textcolor{hwColor}{
        Lastly, the force between Earth and Moon. \\ \\
        $
          \overrightarrow{F_{EM}}=-\overrightarrow{\nabla}V_{EM}=-\dfrac{\partial}{\partial x}\left(G\dfrac{m_E ~ m_M}{x_E-x_M}\right)=-G(m_E ~ m_M)\dfrac{0-1}{(x_E-x_M)^2} ~ \hat{x} \\
          \\
          \overrightarrow{F_{EM}}=G\dfrac{m_E ~ m_M}{(x_E-x_M)^2} ~ \hat{x} \\
        $
      }

      \textcolor{hwColor}{
        $
          \Longrightarrow \begin{cases}
            \overrightarrow{F_{ES}}=-\overrightarrow{F_{SE}} \\
            \\
            \overrightarrow{F_{SM}}=-\overrightarrow{F_{MS}} \\
            \\
            \overrightarrow{F_{EM}}=-\overrightarrow{F_{ME}} \\
          \end{cases}
        $
        \\
        \\
        Therefore, Newton’s third law holds for the system.
      }


      \item Show that the total momentum of the system is conserved.
      
        \textcolor{hwColor}{
          We know that $m$ is constant, hence: \\
          $
            \overrightarrow{P}\equiv m\overrightarrow{V} \\ \\
            \overrightarrow{F}=\dfrac{d\overrightarrow{P}}{dt}=\dfrac{m\overrightarrow{V}}{dt}=m\dfrac{\overrightarrow{V}}{dt}+\overrightarrow{V}\dfrac{m}{dt} \rightarrow \overrightarrow{F}=m\dfrac{\overrightarrow{V}}{dt} \\ \\
          $
          Now that we proved $\overrightarrow{F}=\dfrac{d\overrightarrow{P}}{dt}$, let's find the totoal momentum of the system. \\
          The totoal force acting on the system consists of the sum of all the internal and external forces. Since, we consider this system
          a closed one then we can assume the external force is zero. \\
          \\
          $
            \overrightarrow{F_{Total}}=\overrightarrow{F_{Internal}}+\overrightarrow{F_{External}}=\overrightarrow{F_{Internal}}+0 \\ \\
          $
          It is important to keep in mind that the total internal force is zero because all the internal forces have the same magnitude but they have the opposite direction. 
          Therefore, they all cancel out each other. Hence, we are safe to say, the totoal force of the system is zero. \\
          $
            \dfrac{\overrightarrow{P_{Total}}}{dt}=\overrightarrow{F_{Total}}=0 \Rightarrow \overrightarrow{P_{Total}}=Constant 
          $ \\
          \\
          Hence, the momentum of the system is conserved.
        }

    \end{enumerate}

    \item A ball is thrown straight up at speed $v_0$ in the presence of gravity.
    \begin{enumerate}
      \item Ignoring air resistance at first, what is the maximum height the
      ball reaches?
      \item Now include Newtonian (quadratic) air resistance, with drag coefficient c. Letting the $\hat{y}$ direction point upwards, write down the
      equation of motion (be careful with the signs).

      \item Solve the equation of motion to find $v(t)=(\dot{y(t)})$. Remember to check that $v(0) = v_0$.

      \item Without integrating anything yet, calculate the time at which the ball reaches its maximum height.

      \item Find the formula y(t).(Note that $\dfrac{d}{dx}ln cos(x)=-tan(x)$)

      \item For a baseball thrown upwards at 45 mph (roughly 20 m/s), with g roughly 10 m/s2
      , mass m = 0.14kg, and drag coefficient $c=1.4 × 10^{-3}Ns^2/m^2$
      , calculate the height reached by the ball with and without air resistance.
      
    \end{enumerate}

    \item  A proton of mass m and charge q is in both a uniform electric field,
    $\overrightarrow{E}=E\hat{y}$ and a uniform magnetic field $\overrightarrow{B}=B\hat{y}$. Write down the three
    Cartesian components of the Lorentz force law and solve for the motion of the proton.

  \end{enumerate}

\end{document}
