\documentclass[fleqn]{article}
\oddsidemargin 0.0in
\textwidth 6.0in
\thispagestyle{empty}
\usepackage{import}
\usepackage{amsmath}
\usepackage{graphicx}
\usepackage{flexisym}
\usepackage{amssymb}
\usepackage{bigints} 
\usepackage[english]{babel}
\usepackage[utf8x]{inputenc}
\usepackage{float}
\usepackage[colorinlistoftodos]{todonotes}

\definecolor{hwColor}{HTML}{AD53BA}

\begin{document}

  \begin{titlepage}

    \newcommand{\HRule}{\rule{\linewidth}{0.5mm}}

    \center


    \textsc{\LARGE Arizona State University}\\[1.5cm]

    \textsc{\LARGE Classical Parts/Field/Matter I}\\[1.5cm]


    \begin{figure}
      \includegraphics[width=\linewidth]{asu.png}
    \end{figure}


    \HRule \\[0.4cm]
    { \huge \bfseries Homework Five}\\[0.4cm] 
    \HRule \\[1.5cm]

    \textbf{Behnam Amiri}

    \bigbreak

    \textbf{Prof: Maulik Parikh}

    \bigbreak


    \textbf{{\large \today}\\[2cm]}

    \vfill

  \end{titlepage}

  \begin{enumerate}
    \item \textbf{The Shape of Water}. A round bucket of water is spinning at angular velocity $\omega$ about its axis of symmetry, 
    in the presence of gravity, g. Letting $z_0$ be the height of the water at the center of the bucket, find the height of the water, $z(r)$, as a function of the radius.

      \textcolor{hwColor}{
        \\
        Most of the time when we hear a physical problem has symmetry about, for simplicity it's recommened 
        to use the cylindrical coordinates. We are told that this problem is in the presence of gravity (vertically downward),
        hence we have the centrifugal force acting in the radial direction. From page 344 of the textbook we have the centrifugal force as:
        \\
        \\
        $
          F_{cf}=m \left(\Omega \times r\right) \times \Omega
          \\
          \\
          =mr\Omega^2 sin(\pi/2) \hat{\rho}
          \\
          \\
          \therefore ~~~ F_{cf}=mr \Omega^2 \hat{\rho} ~~~~ \surd
          \\
          \\
          F=F_{cf}-mg \hat{z}=mr \Omega^2 \hat{\rho}-mg \hat{z} ~~~ \surd
        $
        \\
        \\
        Based on the fact that the system is conserved, we have: 
        \\
        \\
        $
          U=-\left[\bigints \left(mr \Omega^2 \hat{\rho}\right) dr+ \bigints \left(-mg \hat{z}\right) dz\right]
          \\
          \\
          =-\left[\dfrac{1}{2}mr^2 \Omega^2-mgz\right]
          \\
          \\
          \\
          \therefore ~~~~ U=-\dfrac{1}{2}mr^2 \Omega^2-mgz
        $
        \\
        \\
        Let's set the potential energy to zero. (since the system is conserved, thus the potential must be a constant) 
        \\
        \\
        $
          U=0 \Longrightarrow -\dfrac{1}{2}mr^2 \Omega^2-mgz=0
          \\
          \\
          \\
          z(r)=\dfrac{r^2 \Omega^2}{2g}+z_0 ~~~ \surd
        $
      }

    \pagebreak

    \item \textbf{Black Hole Spaghetti}. You jump feet first into a black hole. For simplicity, assume you have
    a height $h$ and that the mass of your feet and of your head are about the same, $m$. Assuming the Newtonian gravitational formula to hold
    all the way down to the center of the black hole $(r = 0)$, calculate the tidal force on your head and on your feet as a function of the distance $r$
    to your center of mass (assumed midway between your head and feet). Suppose your body has a breaking tension $T$. Calculate the difference
    in tidal force between your head and feet (keeping track of signs) and show that, regardless of $T$, you will be ripped to pieces as you approach
    the center of the black hole. This process is called spaghettification.

      \textcolor{hwColor}{
        \\
        \\
        We know that we have a body traveling near the black hole. Since we have a very strong gravitational force the
        body will have experience more gravitational force towards the feet. Let's suppose $F_h$ is the force on the head 
        and $F_f$ is the force on the feet. so we have: \\
        \\ 
        $
          F=F_f \left(-\hat{r}\right)-F_h \left(-\hat{r}\right)
          \\
          \\
          \\
          =\left[G\dfrac{Mm}{(r+d)^2}-G\dfrac{Mm}{(r-d)^2}\right] \hat{r}
          \\
          \\
          \\
          =GMm \left(\dfrac{(r-d)^2-(r+d)^2}{(r-d)^2(r+d)^2}\right) \hat{r}
          \\
          \\
          \\
          =GMm \left(\dfrac{-2hr}{(r^2-d^2)^2}\right) \hat{r}, since ~~~ r>>>d
          \\
          \\
          \\
          =2G\dfrac{Mmh}{r^3} \left(-\hat{r}\right) 
        $ 
        \\
        \\
        Let's have $T$ as the breaking force of the body. When the below statement holds the body
        would be destroyed.
        \\
        \\
        $
          T_0< 2G\dfrac{Mmh}{r^3}
        $
        \\
        \\
        As $r$ goes to zero, the tidal force of the body goes to infinity, therefore the body can not tolerate it.
        \\
        \\
        \\
        $
          \therefore ~~~~~ r=\sqrt[3]{\dfrac{T_0}{2GMmh}} ~~~~ \surd
        $
        \\
        \\
        This last formula shows that the body starts to break apart at a distance of $r$.
      }

    \pagebreak

    \item A ball is thrown vertically up to a height h above the surface of the earth at a northern latitude $\lambda$. Ignoring air resistance, show that the
    ball returns to the ground at a point $\dfrac{4}{3} \omega cos(\lambda \sqrt{8h^3/g})$ to the west, where $\omega$ is the angular velocity of the earth.

      \textcolor{hwColor}{
        \\
        We are told that the ball is thrown vertically up, so let's use the Cartesian coordinate. \\
        \\
        $
          \begin{cases}
            u=(0,0,v_0)
            \\
            v=(\dot{x}, \dot{y}, \dot{z})
            \\
            r=(x,y,z)
            \\
            \omega=(0, \omega cos(\phi), \omega sin(\phi))
            \\
            g=(0, 0, -g_0)
          \end{cases}
        $
        \\
        \\
        For this system, we consider two forces:
        \begin{itemize}
          \item The force due to gravity
          \item The Coriolis effect
        \end{itemize}
      }

      \textcolor{hwColor}{
        \\
        Hence we can have the following:
        \\
        \\
        $
          m \ddot{r}=mg+2m (v \times \omega)
          \\
          \\
          \ddot{r}=g+2 (v \times \omega)
          \\
          \\
          \ddot{r}=g+2 \left[(\dot{x}, \dot{y}, \dot{z}) \times (0, \omega cos(\phi), \omega sin(\phi)) \right]
          \\
          \\
          (\ddot{x}, \ddot{y}, \ddot{z})=(0,0,-g_0)+2(\dot{y} \omega sin(\phi)-\dot{z} \omega cos(\phi), -\dot{x} \omega sin(\phi), \dot{x} \omega cos(\phi))
          \\
          \\
          \\
          \therefore ~~~~ \begin{cases}
            \ddot{x}=2 \left[\dot{y} \omega sin(\phi)-\dot{z} \omega cos(\phi)\right]
            \\
            \\
            \ddot{y}=-\dot{x} \omega sin(\phi)
            \\
            \\
            \ddot{z}=-g+\dot{x} \omega cos(\phi)
          \end{cases}
        $
        \\
        \\
        \\
        By assuming $\omega=0$ we have: \\
        \\
        $
          \begin{cases}
            \ddot{x}=0
            \\
            \\
            \ddot{y}=-0
            \\
            \\
            \ddot{z}=-g
          \end{cases}
          \\
          \\
          \therefore ~~~~ \begin{cases}
            x=0
            \\
            y=0
            \\
            z=-\dfrac{1}{2} gt^2+v_0 t
            \\
            \dot{x}=0
            \\
            \dot{y}=0
            \\
            \dot{z}=v_0-gt
          \end{cases}
        $
        \\
        \\
        Let's see what we have so far. Based on the above values we can tell that at $z=0$ the factor t is $2\dfrac{v_0}{g}$ and 
        when $\dot{z}=0$ we must be at the max height ($t=\dfrac{v_0}{g}$). Hence we can conclude:
        \\
        \\
        $
          \begin{cases}
            \ddot{x}=-2 \dot{z} \omega cos(\phi)=2(-v_0+gt) \omega cos(\phi)=2(gt-\sqrt{2gh}) \omega cos(\phi)
            \\
            \ddot{y}=0
            \\
            \ddot{z}=-g
          \end{cases}
          \\
          \\
          \\
          \\
          \therefore ~~~~~ \begin{cases}
            \dot{x}=2 \left[\dfrac{1}{2} gt^2-\sqrt{2ght}\right] \omega cos(\phi) ~~~~ \surd
            \\
            \\
            \\
            x=2 \left[\dfrac{1}{6} gt^3-\dfrac{1}{2} \sqrt{2gh}t^2\right]    ~~~~~~~~~~~ \surd
          \end{cases}
          \\
          \\
          \\
          \Longrightarrow x=2\omega cos(\phi) \left[\dfrac{4}{3} \dfrac{g v^3_0}{g^3}-2\dfrac{v^2_0 \sqrt{2gh}}{g^2}\right]
          \\
          \\
          \\
          \therefore ~~~~ x=-\dfrac{4}{3} \omega cos(\phi) \sqrt{\dfrac{8h^3}{g}} ~~~~~ \surd
        $
        \\
        \\
        Since we have a negative in our result, the ball will fall west. Note that the x direction we choice is East. 
      }

    \pagebreak

    \item An unfortunate ant (m = 1 mg) is on top of a tennis ball when the ball is suddenly served. Assume that, as the ball flies over the net,
    it has an instantaneous velocity of $v=90 ~ kmph$ in the $+\hat{x}$ direction and is slowing down (due to air resistance) at a rate of
    $5 ~ ms^{-2}$. The ball also has topspin, spinning about the $+\hat{y}$ axis at $3000 ~ rpm$. The spin is decreasing (again because of air resistance) at a rate of
    $600 ~ rpm$ per second. The panicked ant, which is on top of the ball at the moment it crosses the net, is running in the $-\hat{x}$ direction with an
    instantaneous velocity of $10 ~ cm ~ s^{-1}$ relative to the ball. Take the radius of the tennis ball to be $3.3 ~ cm$. Calculate the magnitude in Newtons
    (to one significant figure) and Cartesian direction of each of the forces felt by the poor ant (no need to add up all the forces vectorially). You
    can ignore the effect of air resistance on the ant itself as well as exotic phenomena like the Magnus effect which causes a spinning ball to dip.

      \textcolor{hwColor}{
        As we are mouring for the loss and the dark destiny of the poor ant, we are asked to 
        calculate the magnitude in Newtons and Cartesian direction of each of the forces felt 
        by him. \\
        \\
        We have the Earth as our inertial frame. and the ant is in our non-inertial frame with an angular velocity.
        The three types of forces on the ant are Centrifugal, Coriolis, and Euler. 
        \begin{itemize}
          \item $F_{cor}=-2m \left(\omega \times v\right)$
          \item $F_{cen}=-m \times \omega^2 r$
          \item $F_{eul}=-m \dfrac{d \omega}{dt} \times r$
        \end{itemize}
      }

      \textcolor{hwColor}{
        \begin{itemize}
          \item $v_a=-0.1 ~ m/s \hat{i}$ Velocity of the ant without ball.
          \item $v_b=25 ~ m/s \hat{i}$ Velocity of the ball.
          \item $v_{tot}=24.9 ~ m/s \hat{i}$ Velocity of ant with the velocity of the ball.
          \item $a_b=-5 ~ m/s^2 \hat{i}$ Drag acceleration of the ball.
          \item $\omega=-3000 ~ rpm ~ \hat{j}=-100 ~ rad/s \hat{j}$ Angular velocity of the ball.
          \item $\alpha=600 ~ rpm \hat{j}=20 \pi ~ rad/s \hat{j}$ Angular drag acceleration of the ball.
          \item $r==0.033 ~ m \hat{k}$ The ant is on the top of the ball.
          \item $m=10^{-6} ~ kg$ Mass of the poor ant.   
        \end{itemize}
      }

      \textcolor{hwColor}{
        $
          F_{cor}=2m \dot{r} \times \omega
          \\
          \\
          =2m\left(v^{tot} \hat{i} \times \omega \hat{j}\right)
          \\
          \\
          =2m \times 10^{-6} \times 24.9 \times \left(-100 \pi\right) \hat{k}
          \\
          \\
          \therefore ~~~~ F_{cor}=-\dfrac{2}{100} \hat{k} ~ N
          \\
          \\
          \rule{15cm}{1pt}
          \\
          \\
          F_{cf}=m \left(\omega \times r\right) \times \omega
          \\
          \\
          =m \left(\omega \hat{j} \times r \hat{k}\right) \times \omega \hat{j}
          \\
          \\
          =m \left(\omega r \hat{i}\right) \times \omega \hat{j} 
          \\
          \\
          =m \left(\omega^2 r\right) \hat{k}
          \\
          \\
          =10^{-6} \left(-100 \pi\right)^2 \left(0.033\right) \hat{k}
          \\
          \\
          \\
          \therefore ~~~~~ F_{cf}=\dfrac{3}{1000} \hat{k} ~ N 
          \\
          \\
          \rule{15cm}{1pt}
          \\
          \\
          F_{eul}=m \alpha \times r
          \\
          \\
          =m \alpha \hat{j} \times r \hat{k}
          \\
          \\
          =m \alpha r \hat{i}
          \\
          \\
          10^{-6} \times 20 \pi \times 0.033 \hat{i}
          \\
          \\
          \\
          \therefore ~~~~ F_{eul}=2 \times 10^{-6} \hat{i} ~ N
        $
      }

    \pagebreak

    \item A solid cylinder of height $L$ and radius $R$ has uniform mass density $\rho$. Find the moment of inertia tensor about the center of the cylinder. For
    what value of $L/R$ is the cylinder equally easy to spin about any axis?

      \textcolor{hwColor}{
        $
          \mathcal{X}=\bigints \bigints \bigints\limits_{V}=\bigints\limits_{0}^{2\pi} \rho (y^2+z^2) dV
          \\
          \\
          =\bigints\limits_{0}^{2 \pi} \bigints\limits_{-R/2}^{+R/2} \bigints\limits_{0}^{r} \rho r \left(r^2 sin^2(\phi)+z^2\right) dr dz d\phi
          \\
          \\
          =\rho \bigints\limits_{0}^{2 \pi} \bigints\limits_{-R/2}^{+R/2} \left[\dfrac{1}{4} r^4 sin^2(\phi)+\dfrac{1}{2} r^2 z^2\right] dz d\phi
          \\
          \\
          =\rho \bigints\limits_{0}^{2 \pi} \left[\dfrac{1}{4} Rr^4 sin^2(\phi)+\dfrac{1}{24} r^2 R^3\right] d\phi 
          \\
          \\
          =\dfrac{M}{\pi R r^2} \left(\dfrac{1}{4} R r^4 \pi+\dfrac{1}{12} r^2 R^3 \pi\right)
          \\
          \\
          \\
          \therefore ~~~~~ \mathcal{X}=\dfrac{1}{4} Mr^2+\dfrac{1}{12} MR^2
          \\
          \\
          \rule{15cm}{1pt}
          \\
          \\
          \mathcal{Y}=\bigints \bigints \bigints\limits_{V}=\bigints\limits_{0}^{2\pi} \rho (x^2+z^2) dV
          \\
          \\
          =\bigints\limits_{0}^{2 \pi} \bigints\limits_{-R/2}^{+R/2} \bigints\limits_{0}^{r} \rho r \left(r^2 cos^2(\phi)+z^2\right) dr dz d\phi
          \\
          \\
          =\rho \bigints\limits_{0}^{2 \pi} \bigints\limits_{-R/2}^{+R/2} \left[\dfrac{1}{4} r^4 cos^2(\phi)+\dfrac{1}{2} r^2 z^2\right] dz d\phi
          \\
          \\
          =\rho \bigints\limits_{0}^{2 \pi} \left[\dfrac{1}{4} Rr^4 cos^2(\phi)+\dfrac{1}{24} r^2 R^3\right] d\phi 
          \\
          \\
          =\dfrac{M}{\pi R r^2} \left(\dfrac{1}{4} R r^4 \pi+\dfrac{1}{12} r^2 R^3 \pi\right)
          \\
          \\
          \\
          \therefore ~~~~~ \mathcal{Y}=\dfrac{1}{4} Mr^2+\dfrac{1}{12} MR^2
          \\
          \\
          \rule{15cm}{1pt}
          \\
          \\
          \mathcal{Z}=\bigints \bigints \bigints\limits_{V}=\bigints\limits_{0}^{2\pi} \rho (x^2+y^2) dV
          \\
          \\
          =\bigints\limits_{0}^{2 \pi} \bigints\limits_{-R/2}^{+R/2} \bigints\limits_{0}^{r} \rho r \left(r^2 cos^2(\phi)+r^2 sin^2(\phi)\right) dr dz d\phi
          \\
          \\
          =\bigints\limits_{0}^{2 \pi} \bigints\limits_{-R/2}^{+R/2} \bigints\limits_{0}^{r} \rho r^3 dr dz d\phi
          \\
          \\
          =\dfrac{1}{4} \bigints\limits_{0}^{2 \pi} \bigints\limits_{-R/2}^{+R/2} \rho r^4 dr dz d\phi
          \\
          \\
          =\dfrac{1}{2} \pi R r^4 \rho
          \\
          \\
          \\
          \therefore ~~~~~ \mathcal{Z}=\dfrac{1}{2} M r^2
        $
        Hence we have: \\
        \\
        \\
        $\therefore ~~~~~
          \begin{pmatrix}
            \dfrac{1}{4} Mr^2+\dfrac{1}{12} MR^2 & 0 & 0
            \\
            0 & \dfrac{1}{4} Mr^2+\dfrac{1}{12} MR^2 & 0
            \\
            0 & 0 & \dfrac{1}{2} M r^2
          \end{pmatrix}
          \\
          \\
          \dfrac{1}{12} ML^2+\dfrac{1}{4} MR^2=\dfrac{1}{2}MR^2 
          \\
          \\
          \\
          \therefore ~~~~~ \dfrac{L}{R}=\sqrt{3} ~~~~ \surd
        $
      }

    \pagebreak

    \item A pool ball has radius $R$. Find the height, measured from the surface of the pool table, at which the pool ball should be struck with the cue
    stick so that the ball will roll without slipping.

      \textcolor{hwColor}{
        \includegraphics[height=5cm, width=8cm]{Capture.JPG}
        \\
        \\
        For a sphere the moment of inertia is $I=\dfrac{2}{5} m R^2$. The distance from the center of mass
        is $d=(h-R)\hat{k}$. We have: \\
        \\
        $
          \tau=I \dfrac{a_{cx}}{R}=\dfrac{2}{5} m R^2 \dfrac{a_{cx}}{R}
          \\
          \\
          F_{ext} \hat{i} (h-R) \hat{k}=\dfrac{2}{5} mRa_{cx} 
          \\
          \\
          \\
          \therefore ~~~~ F_{ext}(h-R)=\dfrac{2}{5} mRa_{cx} 
        $
        We are told that the ball roll without slipping, hence 
        \\
        \\
        $
          F_{ext}-F_{x}=ma_{cx} \Longrightarrow F_{ext}=ma_{cx}
          \\
          \\
          \\
          \therefore ~~~~~ ma_{cx} (h-R)=\dfrac{2}{5} mRa_{cx} 
          \\
          \\
          h-r=\dfrac{2}{5} r
          \\
          \\
          \\
          \therefore ~~~~ h=\dfrac{7}{5} r
        $
      }

    \pagebreak

    \item On a spinning spherical planet the free-fall acceleration at the North Pole is $g_0$ while at the equator it is $bg_0$ (where $0\leq b\leq 1$). 
    Show that, at latitude $\lambda$, the acceleration is given by $g(\lambda)=g_0 \sqrt{sin^2(\lambda)+b^2 cos^2(\lambda)}$.

      \textcolor{hwColor}{
        We got the free-fall acceleration as:
        \\
        \\
        $
          g=g_0+(\omega \times r) \times \omega=-g_0 \hat{r}+\omega^2 r sin(\phi) \hat{\rho}
        $
        \\
        \\
        \begin{itemize}
          \item $\hat{r}$: Spherical radial vector
          \item $\hat{\rho}$: Cylindrical radial vector
        \end{itemize}
      }

      \textcolor{hwColor}{
        At the equator we have:
        \begin{itemize}
          \item $g=b g_0$
          \item $g=g_0-r \omega^2$
        \end{itemize} 
      }

      \textcolor{hwColor}{
        Based on what we have so far:
        \\
        \\
        $
          g=-g_0 \hat{r}+(1-b) g_0 sin(\phi) \hat{\rho}
          \\
          \\
          =-g_0 \left[cos(\phi) \hat{z}+sin(\phi) \hat{\rho}\right]+(1-b) g_0 sin(\phi) \hat{\rho}
          \\
          \\
          =-g_0 cos(\phi) \hat{z}+\left[(1-b) g_0 sin(\phi)-g_0 sin(\phi)\right] \rho
          \\
          \\
          =-g_0 cos(\phi) \hat{z}-b g_0 sin(\phi) \hat{\rho}
          \\
          \\
          \\
          |g|=\sqrt{\left(-g_0 cos(\phi)\right)^2+\left(b g_0 sin(\phi)\right)^2}
          \\
          \\
          \\
          \therefore ~~~~ |g|=g_0\sqrt{cos^2(\phi)+b^2 sin^2(\phi)} ~~~~ \surd
        $
      }

    \pagebreak

    \item \textbf{Angular Momentum in Different Dimensions}. We are used to thinking of angular momentum as a vector:
    $\overrightarrow{L}=\overrightarrow{r} \times \overrightarrow{p}$. However, the cross product is only defined in three dimensions. Let us
    see if we can figure out what kind of geometric object (scalar, vector,
    matrix, etc) angular momentum really is.
    \begin{itemize}
      \item Begin in two dimensions. Write down the Lagrangian for a free
      particle in polar coordinates and find the conserved quantity due
      to the fact that $\theta$ is a cyclic (or ignorable) coordinate.

        \textcolor{hwColor}{
          $
            L(r, \dot{r}, \dot{\phi})=\dfrac{1}{2} m (\dot{r}^2+r^2 \dot{\phi}^2)
            \\
            \\
            \dfrac{d}{dt} mr^2 \dot{\phi}=0 
            \\
            \\
            \therefore ~~~~ mr^2 \dot{\phi}=Constant
            \\
            \\
            \ddot{r}=r \dot{\phi}^2
          $
          \\
          \\
          We just showed that angular momentum is conserved.
        }

      \item Transforming to Cartesian coordinates, express this angular momentum in terms of Cartesian position
      $x, y$ and momentum $p_x, p_y$ components.

        \textcolor{hwColor}{
          $
            L(\dot{x}, \dot{y})=\dfrac{1}{2}m (\dot{x}^2+\dot{y}^2)
            \\
            \\
            \begin{cases}
              \dfrac{d}{dt}m \dot{x}=0 \Longrightarrow p_x=Constant
              \\
              \\
              \dfrac{d}{dt}m \dot{y}=0 \Longrightarrow p_y=Constant
            \end{cases}
            \\
            \\
            \therefore ~~~~ p=p_x \hat{i}+p_y \hat{j} ~ and ~ r=x \hat{i}+y \hat{j}
            \\
            \\
            \therefore ~~~~ L=r \times p=\left(x p_y-yp_x\right) \hat{k} ~~~ \surd
          $
        }

      \item In three dimensions, write down each of the three components of
      the angular momentum $\overrightarrow{L}=\overrightarrow{r} \times \overrightarrow{p}$ in terms of Cartesian position
      $x,y,z$ and momentum $p_x, p_y, p_z$ components.

        \textcolor{hwColor}{
          $
            L(\dot{x}, \dot{y}, \dot{z})=\dfrac{1}{2}m (\dot{x}^2+\dot{y}^2+\dot{z}^2)
            \\
            \\
            \begin{cases}
              \dfrac{d}{dt}m \dot{x}=0 \Longrightarrow p_x=Constant
              \\
              \\
              \dfrac{d}{dt}m \dot{y}=0 \Longrightarrow p_y=Constant
              \\
              \\
              \dfrac{d}{dt}m \dot{z}=0 \Longrightarrow p_z=Constant
            \end{cases}
            \\
            \\
            \\
            \begin{cases}
              p=p_x \hat{i}+p_y \hat{j}+p_z \hat{k}
              \\
              \\
              r=x \hat{i}+y \hat{j}+z \hat{k}
            \end{cases}
            \\
            \\
            \\
            \therefore ~~~~ L=r \times p=\left(p_y z-p_z y\right) \hat{i}+\left(z p_x-x p_z\right) \hat{j}+\left(x p_y-y p_x\right) \hat{k}
          $
        }

      \item Looking back at your answer for two and three dimensions, try
      to find a pattern. For 4 dimensions of space, list the conserved
      angular momenta in terms of Cartesian position w, x, y, z and momentum $p_{\omega}, p_x, p_y,p_z$ components.

        \textcolor{hwColor}{
          $
          L(\dot{x}, \dot{y}, \dot{z}, \dot{w})=\dfrac{1}{2}m (\dot{x}^2+\dot{y}^2+\dot{z}^2+\dot{w}^2)
          \\
          \\
          \begin{cases}
            \dfrac{d}{dt}m \dot{x}=0 \Longrightarrow p_x=Constant
            \\
            \\
            \dfrac{d}{dt}m \dot{y}=0 \Longrightarrow p_y=Constant
            \\
            \\
            \dfrac{d}{dt}m \dot{z}=0 \Longrightarrow p_z=Constant
            \\
            \\
            \dfrac{d}{dt}m \dot{w}=0 \Longrightarrow p_w=Constant
          \end{cases}
          $
          \\
          \\
          The angular momentum can be written as an asymmetric matrix. 
          \\
          \\
          $
            L_4=\begin{pmatrix}
              L_{xx} & L_{xy} & L_{xz} & L_{xw} 
              \\
              L_{yx} & L_{yy} & L_{yz} & L_{yw}
              \\
              L_{zx} & L_{zy} & L_{zz} & L_{zw}
              \\
              L_{wx} & L_{wy} & L_{wz} & L_{ww}
            \end{pmatrix}
            \\
            \\
            \\
            \\
            =\begin{pmatrix}
              0 & xp_y-yp_x & xp_z-zp_x &  xp_w-wp_x
              \\
              yp_x-xp_y & 0 & yp_z-zp_y & yp_w-wp_y 
              \\
              zp_x-xp_z & zp_y-yp_z & 0 & zp_w-wp_z
              \\
              wp_x-xp_w & wp_y-yp_w & wp_z-zp_w & 0
            \end{pmatrix}
            \\
            \\
            \\
          $
          We see there are 6 independent components to angular momentum.
        }
  
      \item In general, if there are $d$ dimensions of space, how many components of angular momentum are there?

        \textcolor{hwColor}{
          \begin{itemize}
            \item For 0 and 1 dimension of space, we have zero components.
            \item For 2 dimensions, we have one component.
            \item For three dimensions, we have 3 components.
            \item For four dimensions, we have 6 components.
          \end{itemize}
        }

        \textcolor{hwColor}{
          Based on our observation, the pattern we can see is as the following:
          \\
          \\
          \\
          Number of components$=\dfrac{1}{2}\left(d^2-d\right)$.
        }

      \item The number of components of angular momentum clearly does not match the number of components of a vector in $d$ dimensions
      (which is just $d$) except when $d=3$. From the number of independent components, what kind of geometric object (i.e. scalar,
      vector, matrix) is angular momentum?
      Hint: it’s a kind of $d \times d$ matrix. (Incidentally, the same is true for the magnetic field. Only 
      in $d=3$ does the magnetic field have the same number of components as a vector. In general, the magnetic field is a matrix.)
    
        \textcolor{hwColor}{
          \\
          \\
          Angular momentum is an anti-symmetric tensor. For 3-D the angular momentum tensor looks like:
          $
            L_3=\begin{pmatrix}
              0 & xp_y-yp_x & xp_z-zp_x
              \\
              yp_x-xp_y & 0 & yp_z-zp_y 
              \\
              zp_x-xp_z & zp_y-yp_z & 0
            \end{pmatrix}
          $ 
          \\
          \\
          \\
          For 2-D we have: \\
          \\
          \\
          $
            L_2=\begin{pmatrix}
              0 & xp_y-yp_x
              \\
              yp_x-xp_y & 0 
            \end{pmatrix}
          $
          \\
          \\
          For which we only have one component that is independent, implying that this may not even be a vector in
          the z direction, but actually a scalar that tells us the magnitude of the angular momentum and the sign 
          tells us which direction its acting (clockwise/counterclockwise) in.
        }

    \end{itemize}
    
  \end{enumerate}

\end{document}
