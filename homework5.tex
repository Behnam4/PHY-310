\documentclass[fleqn]{article}
\oddsidemargin 0.0in
\textwidth 6.0in
\thispagestyle{empty}
\usepackage{import}
\usepackage{amsmath}
\usepackage{graphicx}
\usepackage{flexisym}
\usepackage{amssymb}
\usepackage{bigints} 
\usepackage[english]{babel}
\usepackage[utf8x]{inputenc}
\usepackage{float}
\usepackage[colorinlistoftodos]{todonotes}

\definecolor{hwColor}{HTML}{AD53BA}

\begin{document}

  \begin{titlepage}

    \newcommand{\HRule}{\rule{\linewidth}{0.5mm}}

    \center


    \textsc{\LARGE Arizona State University}\\[1.5cm]

    \textsc{\LARGE Classical Parts/Field/Matter I}\\[1.5cm]


    \begin{figure}
      \includegraphics[width=\linewidth]{asu.png}
    \end{figure}


    \HRule \\[0.4cm]
    { \huge \bfseries Homework Five}\\[0.4cm] 
    \HRule \\[1.5cm]

    \textbf{Behnam Amiri}

    \bigbreak

    \textbf{Prof: Maulik Parikh}

    \bigbreak


    \textbf{{\large \today}\\[2cm]}

    \vfill

  \end{titlepage}

  \begin{enumerate}
    \item \textbf{The Shape of Water}. A round bucket of water is spinning at angular velocity $\omega$ about its axis of symmetry, 
    in the presence of gravity, g. Letting $z_0$ be the height of the water at the center of the bucket, find the height of the water, $z(r)$, as a function of the radius.

    \item \textbf{Black Hole Spaghetti}. You jump feet first into a black hole. For simplicity, assume you have
    a height $h$ and that the mass of your feet and of your head are about the same, $m$. Assuming the Newtonian gravitational formula to hold
    all the way down to the center of the black hole $(r = 0)$, calculate the tidal force on your head and on your feet as a function of the distance $r$
    to your center of mass (assumed midway between your head and feet). Suppose your body has a breaking tension $T$. Calculate the difference
    in tidal force between your head and feet (keeping track of signs) and show that, regardless of $T$, you will be ripped to pieces as you approach
    the center of the black hole. This process is called spaghettification.

    \item A ball is thrown vertically up to a height h above the surface of the earth at a northern latitude $\lambda$. Ignoring air resistance, show that the
    ball returns to the ground at a point $\dfrac{4}{3} \omega cos(\lambda \sqrt{8h^3/g})$ to the west, where $\omega$ is the angular velocity of the earth.


    \item An unfortunate ant (m = 1 mg) is on top of a tennis ball when the ball is suddenly served. Assume that, as the ball flies over the net,
    it has an instantaneous velocity of $v=90 ~ kmph$ in the $+\hat{x}$ direction and is slowing down (due to air resistance) at a rate of
    $5 ~ ms^{-2}$. The ball also has topspin, spinning about the $+\hat{y}$ axis at $3000 ~ rpm$. The spin is decreasing (again because of air resistance) at a rate of
    $600 ~ rpm$ per second. The panicked ant, which is on top of the ball at the moment it crosses the net, is running in the $-\hat{x}$ direction with an
    instantaneous velocity of $10 ~ cm ~ s^{-1}$ relative to the ball. Take the radius of the tennis ball to be $3.3 ~ cm$. Calculate the magnitude in Newtons
    (to one significant figure) and Cartesian direction of each of the forces felt by the poor ant (no need to add up all the forces vectorially). You
    can ignore the effect of air resistance on the ant itself as well as exotic phenomena like the Magnus effect which causes a spinning ball to dip.


    \item A solid cylinder of height $L$ and radius $R$ has uniform mass density $\rho$. Find the moment of inertia tensor about the center of the cylinder. For
    what value of $L/R$ is the cylinder equally easy to spin about any axis?


    \item A pool ball has radius $R$. Find the height, measured from the surface of the pool table, at which the pool ball should be struck with the cue
    stick so that the ball will roll without slipping.


    \item On a spinning spherical planet the free-fall acceleration at the North Pole is $g_0$ while at the equator it is $bg_0$ (where $0\leq b\leq 1$). 
    Show that, at latitude $\lambda$, the acceleration is given by $g(\lambda)=g_0 \sqrt{sin^2(\lambda)+b^2 cos^2(\lambda)}$.


    \item \textbf{Angular Momentum in Different Dimensions}. We are used to thinking of angular momentum as a vector:
    $\overrightarrow{L}=\overrightarrow{r} \times \overrightarrow{p}$. However, the cross product is only defined in three dimensions. Let us
    see if we can figure out what kind of geometric object (scalar, vector,
    matrix, etc) angular momentum really is.
    \begin{itemize}
      \item Begin in two dimensions. Write down the Lagrangian for a free
      particle in polar coordinates and find the conserved quantity due
      to the fact that $\theta$ is a cyclic (or ignorable) coordinate.

      \item Transforming to Cartesian coordinates, express this angular momentum in terms of Cartesian position
      $x, y$ and momentum $p_x, p_y$ components.

      \item In three dimensions, write down each of the three components of
      the angular momentum $\overrightarrow{L}=\overrightarrow{r} \times \overrightarrow{p}$ in terms of Cartesian position
      $x,y,z$ and momentum $p_x, p_y, p_z$ components.

      \item Looking back at your answer for two and three dimensions, try
      to find a pattern. For 4 dimensions of space, list the conserved
      angular momenta in terms of Cartesian position w, x, y, z and momentum $p_{\omega}, p_x, p_y,p_z$ components.

      \item In general, if there are $d$ dimensions of space, how many components of angular momentum are there?


      \item The number of components of angular momentum clearly does not match the number of components of a vector in $d$ dimensions
      (which is just $d$) except when $d=3$. From the number of independent components, what kind of geometric object (i.e. scalar,
      vector, matrix) is angular momentum?
      Hint: it’s a kind of $d \times d$ matrix. (Incidentally, the same is true for the magnetic field. Only 
      in $d=3$ does the magnetic field have the same number of components as a vector. In general, the magnetic field is a matrix.)
  
    \end{itemize}
    
  \end{enumerate}

\end{document}
