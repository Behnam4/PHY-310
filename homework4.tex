\documentclass[fleqn]{article}
\oddsidemargin 0.0in
\textwidth 6.0in
\thispagestyle{empty}
\usepackage{import}
\usepackage{amsmath}
\usepackage{graphicx}
\usepackage{flexisym}
\usepackage{amssymb}
\usepackage{bigints} 
\usepackage[english]{babel}
\usepackage[utf8x]{inputenc}
\usepackage{float}
\usepackage[colorinlistoftodos]{todonotes}

\definecolor{hwColor}{HTML}{AD53BA}

\begin{document}

  \begin{titlepage}

    \newcommand{\HRule}{\rule{\linewidth}{0.5mm}}

    \center


    \textsc{\LARGE Arizona State University}\\[1.5cm]

    \textsc{\LARGE Classical Parts/Field/Matter I}\\[1.5cm]


    \begin{figure}
      \includegraphics[width=\linewidth]{asu.png}
    \end{figure}


    \HRule \\[0.4cm]
    { \huge \bfseries Homework Four}\\[0.4cm] 
    \HRule \\[1.5cm]

    \textbf{Behnam Amiri}

    \bigbreak

    \textbf{Prof: Maulik Parikh}

    \bigbreak


    \textbf{{\large \today}\\[2cm]}

    \vfill

  \end{titlepage}

  \begin{enumerate}
    \item Halley’s comet, observed by Chinese astronomers over $2000$ years ago,
    has an eccentricity of $0.967$ and a period of $76$ years. What shape is its
    orbit? What are its minimum and maximum distances from the sun?
    (Mass of the sun$=2 \times 10^{30} ~ kg$.)

      \textcolor{hwColor}{
        The eccentricity is $0.967$ so the shape must be an ellipse. \\
        \\
        $
          T^2=\dfrac{4 \pi^2 a^3}{G ~ M_s}=\dfrac{4 \pi^2 a^3}{6.67 \times 10^{-11} ~ N.m^2/kg^2 ~ \times 1.98 \times 10^{30} ~ kg} \\
          \\
          \\
          =\dfrac{4 \pi^2}{ 6.67 \times 10^{-11} \times 2 \times 10^{30}} \dfrac{m^3}{ \left(6.67 \times 10^{-12}\right)^3 AU^3} \dfrac{s^2}{m^3} \dfrac{year^2}{\left(3.15 \times 10^7\right)^2 s^2} \\
          \\
          \\
          \therefore ~~~ T^2=0.99a^3 ~~ \surd \\
          \\
          \\
          \Longrightarrow a=\sqrt[3]{T^2}=\sqrt[3]{76^2} \Longrightarrow a \approx 2.7 \times 10^{12} ~~ m ~~~ \surd \\
          \\
          \\
          \begin{cases}
            d_{max}=\dfrac{17.94(1-0.96^2)}{1-\epsilon} \approx 5.28 \times 10^{12} ~~ m \\
            \\
            \\
            d_{min}=\dfrac{17.94(1-0.96^2)}{1+\epsilon}=8.86 \times 10^{10} ~~ m
          \end{cases}
        $
      }
    
    \item A particle of mass $m$ in one dimension is subject to a force
    $$F=\dfrac{a}{x^2} e^{-bt}$$
    where a and b are positive, dimensionful constants. Write down the Lagrangian, 
    the Hamiltonian, the Euler-Lagrange equations, and Hamilton’s equations. Is energy 
    conserved during the motion? Why, or why not?

      \textcolor{hwColor}{
        $
          U(x)=-W=-\bigints\limits_{0}^{x} F(x^') dx^'=-\bigints\limits_{0}^{x} \dfrac{a e^{-bt}}{(x^')^2} dx^' \\
          \\
          \\
          \therefore ~~~ U(x)=\dfrac{e^{-bt} a}{x} \\
          \\
          \\
          \\
          \mathcal{L}(x, \dot{x})=KE-PE=\dfrac{1}{2}m \dot{x}^2-\dfrac{e^{-bt} a}{x} ~~~ \surd \\
          \\
          \\
          \dfrac{ \partial \mathcal{L}}{\partial x}=\dfrac{d}{dt} \dfrac{\partial \mathcal{L}}{\dot{x}}
          \Longrightarrow 0-\left[-\dfrac{a e^{-bt}}{x^2}\right]=\dfrac{d}{dt}\left(m \dot{x}\right) 
          \Longrightarrow \dfrac{a e^{-bt}}{x^2}=m\ddot{x} ~~~~ \surd
        $
        \\
        \\
        The Hamiltonian: \\
        \\
        $
          \mathcal{H}=p\dot{q}-\mathcal{L}=p\dot{q}-\left(KE-PE\right) \\
          \\
          \\
          =p_x \dot{x}-\dfrac{1}{2} m \dot{x}^2+\dfrac{a e^{-bt}}{x}=p_x (\dfrac{p_x}{m})-\dfrac{1}{2} m (\dfrac{p_x}{m})^2+\dfrac{a e^{-bt}}{x}
          \\
          \\
          \\
          \Longrightarrow \begin{cases}
            \mathcal{H}=\dfrac{p^2_x}{2m}+\dfrac{a e^{-bt}}{x} \\
            \\
            \dot{p}_x=-\dfrac{\partial \mathcal{H}}{\partial x}=\dfrac{a e^{-bt}}{x^2} ~~~~~~~~~~~~~~~~~ \surd \\
            \\
            \dot{x}=\dfrac{\partial \mathcal{H}}{\partial p_x}=\dfrac{p_x}{m}
          \end{cases} \\
          \\
          \\
          \\
          \begin{cases}
            KE=\dfrac{1}{2}mv^2 \\
            \\
            dKE=m (v.\dot{v})=F.r \\
            \\
            PE=\dfrac{a e^{-bt}}{x} \\
            \\
            dPE=\dfrac{\partial PE}{\partial t} dt-F.r \\
            \\
            d(KE+PE)=\dfrac{\partial PE}{\partial t}dt
          \end{cases}
        $ \\
        \\
        \\
        As we shown, the change in mechanical energy is not constant, hence the mechanical energy is not conserved.
        The total energy though is conserved. 
      }

    \pagebreak

    \item Suppose that the sun’s mass suddenly decreases by half (of course that
    can’t really happen, since mass-energy is conserved). Assume the earth
    is initially on a circular orbit. What orbit will the earth now have? Will
    it escape the solar system?

      \textcolor{hwColor}{
        Earth's annual pilgrimage around the Sun isn't perfectly circular, 
        but it's pretty close.The present eccentricity of Earth is $\approx 0.016$, hence we can 
        consider its value as zero. \\
        \\
        $
          F=ma_c \Rightarrow m\dfrac{v^2_0}{r}=-\dfrac{G M m}{r^2}, ~~~ T_0=\dfrac{1}{2}mv^2_0 \\
          \\
          \dfrac{T_0}{2}=-\dfrac{G Mm}{r} \\ \\
          \therefore ~~~ T_0=-\dfrac{G M m}{2r} ~~~ \surd
        $
        \\
        \\
        When the sun’s mass suddenly decreases by half, the potential energy changes, but the kinetic energy stays 
        the same. \\
        \\
        $
          E=T_0+\dfrac{G M m}{2r} \\
          \\
          \therefore ~~~~ E=0 ~~~ \surd
        $ \\
        \\
        \\
        Let's find the new escape velocity.Work done in taking the body against gravitational attraction is given by: \\
        \\
        $
          dW=F dx=\dfrac{G Mm}{x^2} dx \\
          \\
          \\
          W=\bigints\limits_{r}^{\infty} dW=\bigints\limits_{r}^{\infty} \dfrac{G Mm}{x^2} dx=GMm \bigints\limits_{r}^{\infty} \dfrac{1}{x^2} dx
          =GMm \bigints\limits_{r}^{\infty} \left(\dfrac{-1}{x}\right) \Big|_{r}^{\infty} \\
          \\
          \\
          \therefore ~~~ W=\dfrac{GMm}{r}
        $ 
        \\
        \\
        By equating kinetic energy to gravitational potential energy. \\
        \\
        \\
        $
          KE=W, \Longrightarrow \dfrac{1}{2} mv^2_e=\dfrac{GMm}{r} \\
          \\
          \\
          \therefore ~~~ v_e=\sqrt{\dfrac{2GM}{r}} ~~ \\ \\
          \\
          M^'_{sun}=\dfrac{1}{2} M_{sun} \Longrightarrow  v_e=\sqrt{\dfrac{(6.67 \times 10^{-11}) (1 \times 10^{30})}{1.5 \times 10^{11}}} \\
          \\
          \\
          v_e=29.80 ~~~ km/s
        $
        \\
        \\
        As we know the Earth is moving around the sun at a speed of nearly $30 ~ km/s$. We see that the Earth will leave 
        the solar system because the escape velocity of earth is now the speed at which earth is traveling.
        Earth did not change speed as its kinetic energy remained constant. Since the Earth will leave the solar 
        system the new orbit will be parabolic.
      }

    \item A particle of mass $m$, subject to gravity, is constrained to move along
    the spiral $z=b\theta, ~ r=r_0$, where $b$ and $r_0$ are positive constants, and $z$
    is the vertical direction. Find Hamilton’s equations. 

      \textcolor{hwColor}{
        Cylindrical coordinate system: \\
        \\
        $
          v=\dot{r} \hat{r}+r \dot{\theta} \hat{\theta}+\dot{z} \hat{z} \Longrightarrow v=b \dot{\theta} \hat{z}+r \dot{\theta} \hat{\theta} \\
          \\
          \\
          \begin{cases}
            PE=mgz \\
            \\
            KE=\dfrac{1}{2}m v^2=\dfrac{1}{2}m v.v=\dfrac{1}{2}m \left[(0, ~ r \dot{\theta}, ~ b \dot{\theta}).(0, ~ r \dot{\theta}, ~ b \dot{\theta})\right]
            =\dfrac{1}{2}m \dot{\theta}^2 (r^2+b^2)
          \end{cases}
        $ \\
        \\
        \\
        \\
        The generalized momentum: \\
        \\
        $
          p=\dfrac{\partial \mathcal{L}}{\partial \dot{\theta}}
          =\dfrac{\partial \left[  \dfrac{1}{2}m(r^2+b^2) \dot{\theta}^2+mgb\theta\right]}{\partial \dot{\theta}} \\
          \Longrightarrow \dot{\theta}=\dfrac{p_{\theta}}{m(r^2+b^2)}
        $
        \\
        \\
        Let's find the Hamiltonian: \\
        \\
        $
          \mathcal{H}=p\dot{q}-\mathcal{L}=p^2_{\theta} \dot{\theta}-KE+PE \\
          \\
          \\
          =\dfrac{p^2_{\theta}}{m(r^2+b^2)}-\dfrac{1}{2}m(r^2+b^2) \dot{\theta}^2+mgb\theta
          =\dfrac{p^2_{\theta}}{m(r^2+b^2)}-\dfrac{1}{2}m(r^2+b^2) \left(\dfrac{p_{\theta}}{m(r^2+b^2)}\right)^2+mgb\theta \\
          \\
          \\
          \\
          \therefore ~~~ \mathcal{H}=mgb\theta+\dfrac{p^2_{\theta}}{2m(r^2+b^2)} ~~~~~ \surd \\
          \\
          \\
          \begin{cases}
            \dot{q}=\dot{\theta}=\dfrac{\partial \mathcal{H}}{\partial p_{\theta}}=\dfrac{\partial}{\partial p_{\theta}} \left[mgb\theta+\dfrac{p^2_{\theta}}{2m(r^2+b^2)}\right] \\
            \\
            \dot{p}=\dot{p}_{\theta}=-\dfrac{\partial \mathcal{H}}{\partial p_{\theta}}=-\dfrac{\partial}{\partial p_{\theta}} \left[mgb\theta+\dfrac{p^2_{\theta}}{2m(r^2+b^2)}\right]
          \end{cases}
          \Longrightarrow \begin{cases}
            \dot{q}=\dot{\theta}=\dfrac{p_{\theta}}{m(r^2+b^2)} \\ 
            \\
            \dot{p}=\dot{p}_{\theta}=-mgb
          \end{cases}
        $ 
      }

    \item Using Cartesian coordinates, show that when $\epsilon > 1$, the equation
    $r(\phi)=\dfrac{c}{1+ \epsilon ~ cos(\phi)}$ can be rewritten as the equation for a shifted hyperbola
    $$\dfrac{(x-\delta)^2}{\alpha^2}-\dfrac{y^2}{\beta^2}=1$$
    Find $\alpha, \beta,$ and $\delta$ in terms of $c$ and $\epsilon$.

      \textcolor{hwColor}{
        We are told $\epsilon > 1$: \\
        \\
        \\
        $
          r(\phi)=\dfrac{c}{1+ \epsilon ~ cos(\phi)} \Rightarrow r(1+ \epsilon ~ cos(\phi))=1 \\
          \\
          \\
        $
        We know $x=r cos(\phi)$, hence:
        $
          r(1+ \epsilon ~ \dfrac{x}{r})=1, ~~~ k \equiv r+\epsilon ~ x
        $
        \\
        \\
        $
          x^2+y^2=r^2=(k- \epsilon x)^2 \\
          \\
          x^2+y^2=k^2-2k \epsilon x+ \epsilon^2 x^2 \\
          \\
          x^2+y^2+2k \epsilon x- \epsilon^2 x^2=k^2 \\
          \\
          x^2(1- \epsilon^2)+y^2+2k \epsilon x=k^2 \\
          \\
          x^2(1- \epsilon^2)^2+y^2(1- \epsilon^2)+2k \epsilon x(1- \epsilon^2)=k^2(1- \epsilon^2) \\
          \\
          \\
          \left[x(1-\epsilon^2)+ \epsilon k\right]^2-y^2|\epsilon^2-1|=k^2 \\
          \\
          \\
          (1- \epsilon^2)^2 \left[x+\dfrac{\epsilon k}{(1- \epsilon^2)}\right]^2-y^2 |\epsilon^2-1|=k^2 \\
          \\
          \\
          \\
          \therefore ~~~~ \left[\dfrac{1-\epsilon^2}{k}\right]^2 \left[\dfrac{\epsilon k}{(1- \epsilon^2)^2}+x\right]^2-y^2 \left[\dfrac{|e^2-1|}{k^2}\right] \\
          \\
          \\
          \\
          \therefore ~~~ \begin{cases}
            \alpha=\dfrac{k}{1-\epsilon^2} \\
            \\
            \beta=\dfrac{k}{\sqrt{|\epsilon^2-1|}} \\
            \\
            \delta=-\dfrac{\epsilon k}{(1- \epsilon^2)}
          \end{cases}
        $
      }


    \item Verify that $r=a cos(\phi)$ is a solution for the central force 
    $\overrightarrow{F}=-\dfrac{k}{r^5} \hat{r}$. Find $a$ in terms of $k, ~ \mu$ 
    (the reduced mass), and $\ell$ (the angular momentum).

    \textcolor{hwColor}{
      Here
    }


    \item \begin{enumerate}
      \item Using the effective potential, show that for a force $\overrightarrow{F}=-\dfrac{a}{r} \hat{r}$,
      there are only bounded orbits, and find the radius of the circular orbit.

        \textcolor{hwColor}{
          Here
        }

      \item Use Kepler’s third law to calculate the altitude above the earth’s
      surface of geostationary satellites, which are east-moving satellites
      over the equator that take exactly one day to go around the earth
      and therefore are always above the same spot. (The mass of the
      earth is $6 \times 10^{24} ~ kg$ and the radius is $6400 ~ km$.)

        \textcolor{hwColor}{
          Here
        }


    \end{enumerate}

  \end{enumerate}

\end{document}
