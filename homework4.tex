\documentclass[fleqn]{article}
\oddsidemargin 0.0in
\textwidth 6.0in
\thispagestyle{empty}
\usepackage{import}
\usepackage{amsmath}
\usepackage{graphicx}
\usepackage{flexisym}
\usepackage{amssymb}
\usepackage{bigints} 
\usepackage[english]{babel}
\usepackage[utf8x]{inputenc}
\usepackage{float}
\usepackage[colorinlistoftodos]{todonotes}

\definecolor{hwColor}{HTML}{AD53BA}

\begin{document}

  \begin{titlepage}

    \newcommand{\HRule}{\rule{\linewidth}{0.5mm}}

    \center


    \textsc{\LARGE Arizona State University}\\[1.5cm]

    \textsc{\LARGE Classical Parts/Field/Matter I}\\[1.5cm]


    \begin{figure}
      \includegraphics[width=\linewidth]{asu.png}
    \end{figure}


    \HRule \\[0.4cm]
    { \huge \bfseries Homework Four}\\[0.4cm] 
    \HRule \\[1.5cm]

    \textbf{Behnam Amiri}

    \bigbreak

    \textbf{Prof: Maulik Parikh}

    \bigbreak


    \textbf{{\large \today}\\[2cm]}

    \vfill

  \end{titlepage}

  \begin{enumerate}
    \item Halley’s comet, observed by Chinese astronomers over $2000$ years ago,
    has an eccentricity of $0.967$ and a period of $76$ years. What shape is its
    orbit? What are its minimum and maximum distances from the sun?
    (Mass of the sun$=2 \times 10^{30} ~ kg$.)

      \textcolor{hwColor}{
        The eccentricity is $0.967$ so the shape must be an ellipse. \\
        \\
        $
          T^2=\dfrac{4 \pi^2 a^3}{G ~ M_s}=\dfrac{4 \pi^2 a^3}{6.67 \times 10^{-11} ~ N.m^2/kg^2 ~ \times 1.98 \times 10^{30} ~ kg} \\
          \\
          \\
          =\dfrac{4 \pi^2}{ 6.67 \times 10^{-11} \times 2 \times 10^{30}} \dfrac{m^3}{ \left(6.67 \times 10^{-12}\right)^3 AU^3} \dfrac{s^2}{m^3} \dfrac{year^2}{\left(3.15 \times 10^7\right)^2 s^2} \\
          \\
          \\
          \therefore ~~~ T^2=0.99a^3 ~~ \surd \\
          \\
          \\
          \Longrightarrow a=\sqrt[3]{T^2}=\sqrt[3]{76^2} \Longrightarrow a \approx 2.7 \times 10^{12} ~~ m ~~~ \surd \\
          \\
          \\
          \begin{cases}
            d_{max}=\dfrac{17.94(1-0.96^2)}{1-\epsilon} \approx 5.28 \times 10^{12} ~~ m \\
            \\
            \\
            d_{min}=\dfrac{17.94(1-0.96^2)}{1+\epsilon}=8.86 \times 10^{10} ~~ m
          \end{cases}
        $
      }
    
    \item A particle of mass $m$ in one dimension is subject to a force
    $$F=\dfrac{a}{x^2} e^{-bt}$$
    where a and b are positive, dimensionful constants. Write down the Lagrangian, 
    the Hamiltonian, the Euler-Lagrange equations, and Hamilton’s equations. Is energy 
    conserved during the motion? Why, or why not?

      \textcolor{hwColor}{
        Hey
      }

    \item Suppose that the sun’s mass suddenly decreases by half (of course that
    can’t really happen, since mass-energy is conserved). Assume the earth
    is initially on a circular orbit. What orbit will the earth now have? Will
    it escape the solar system?

      \textcolor{hwColor}{
        Hey
      }

    \item A particle of mass m, subject to gravity, is constrained to move along
    the spiral $z=b\theta, ~ r=r_0$, where $b$ and $r_0$ are positive constants, and $z$
    is the vertical direction. Find Hamilton’s equations. 

      \textcolor{hwColor}{
        Hey
      }

    \item Using Cartesian coordinates, show that when $\epsilon > 1$, the equation
    $r(\phi)=\dfrac{c}{1+ \epsilon ~ cos(\phi)}$ can be rewritten as the equation for a shifted hyperbola
    $$\dfrac{(x-\delta)^2}{\alpha^2}-\dfrac{y^2}{\beta^2}=1$$
    Find $\alpha, \beta,$ and $\delta$ in terms of $c$ and $\epsilon$.

      \textcolor{hwColor}{
        Hey
      }


    \item Verify that $r=a cos(\phi)$ is a solution for the central force 
    $\overrightarrow{F}=-\dfrac{k}{r^5} \hat{r}$. Find $a$ in terms of $k, ~ \mu$ 
    (the reduced mass), and $\ell$ (the angular momentum).

    \textcolor{hwColor}{
      Hey
    }


    \item \begin{enumerate}
      \item Using the effective potential, show that for a force $\overrightarrow{F}=-\dfrac{a}{r} \hat{r}$,
      there are only bounded orbits, and find the radius of the circular orbit.

        \textcolor{hwColor}{
          Hey
        }

      \item Use Kepler’s third law to calculate the altitude above the earth’s
      surface of geostationary satellites, which are east-moving satellites
      over the equator that take exactly one day to go around the earth
      and therefore are always above the same spot. (The mass of the
      earth is $6 \times 10^{24} ~ kg$ and the radius is $6400 ~ km$.)

        \textcolor{hwColor}{
          Hey
        }


    \end{enumerate}

  \end{enumerate}

\end{document}
